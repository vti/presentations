\documentclass[14pt]{beamer}
\usepackage{listings}
\usepackage[T2A]{fontenc}
\usepackage[utf8]{inputenc}
\usepackage[english,russian]{babel}

\begin{document}

\title{Design by Contract в Perl}
\author{Вячеслав Тихановский (vti)}
\date{2012-12-22}
\setbeamercolor{background canvas}{bg=black,fg=white}
\setbeamercolor{frametitle}{fg=white}
\setbeamercolor{normal text}{bg=black}
\setbeamercolor{whitetext}{fg=white}
\usebeamercolor[fg]{whitetext}

\lstset{commentstyle=\textit}
\lstset{basicstyle=\small,language=Perl}

\frame{\titlepage}

\begin{frame}
    \begin{center}
        showmetheco.de
    \end{center}
\end{frame}

\begin{frame}
    \begin{center}
        ООП
    \end{center}
\end{frame}

\begin{frame}
    \begin{center}
        Moose, Moos, Moo, Mo, M?
    \end{center}
\end{frame}

\begin{frame}
    \begin{center}
        Интерфейсы и наследование от абстрактного класса
    \end{center}
\end{frame}

\begin{frame}[fragile]
    \begin{lstlisting}
package AbstractRenderer;

sub new {...}

sub render { die 'Implement me!' }
    \end{lstlisting}
\end{frame}

\begin{frame}[fragile]
    \begin{lstlisting}
package TT;
use base 'AbstractRenderer';
sub render {
    my $self = shift;
    my ($template, $layout, %vars) = @_;

    return \$rendered;
}

package Caml;
use base 'AbstractRenderer';
sub render {
    my $self = shift;
    my ($template, %vars) = @_;

    return $rendered;
}
    \end{lstlisting}
\end{frame}

\begin{frame}
    \begin{center}
        Design by Contract
    \end{center}
\end{frame}

\begin{frame}
    \begin{center}
        \begin{itemize}
            \item Проверка входных и выходных условий
            \pause
            \item Следование принципу подстановки Лисков (L в SOLID) при
                наследовании
            \pause
            \item Документирование интерфейса
        \end{itemize}
    \end{center}
\end{frame}

\begin{frame}[fragile]
    \begin{lstlisting}
package AbstractRenderer;
sub new {...}

# Requires(STRING, %HASH_OF_ANY_VALUES)
# Ensures(STRING)
# Throws(Exception::TemplateNotFound)
sub render {
    my $self = shift;
    my ($template, %vars) = @_;

    ...

    return $rendered;
}
    \end{lstlisting}
\end{frame}

\begin{frame}[fragile]
    \begin{lstlisting}
package TT;
use base 'AbstractRenderer';
sub render { ... }

package Caml;
use base 'AbstractRenderer';
sub render { ... }
    \end{lstlisting}
\end{frame}

\begin{frame}
    \begin{center}
        \begin{itemize}
            \item Не должно снижать читаемость
            \pause
            \item Не требует больших вычислительных ресурсов
            \pause
            \item Отключаемо
            \pause
            \item Наследуемо
            \pause
            \item Запрещает менять контракт в дочерних классах
            \pause
            \item Понятные ошибки
            \pause
            \item Как можно меньше магии*
            \pause
            \item Похоже на Perl*
        \end{itemize}
    \end{center}
\end{frame}

\begin{frame}
    \begin{center}
        Class::Contract
    \end{center}
\end{frame}

\begin{frame}[fragile]
    \begin{lstlisting}
use Class::Contract;

contract {
  inherits 'BaseClass';

  ctor 'new';

  method 'methodname';
    pre  { ... };
      failmsg 'Error message';

    post  { ... };
      failmsg 'Error message';

    impl { ... };
};
    \end{lstlisting}
\end{frame}

\begin{frame}
    \begin{center}
        Sub::Contract
    \end{center}
\end{frame}

\begin{frame}[fragile]
    \begin{lstlisting}
use Sub::Contract qw(contract);

contract('surface')
    ->in(\&is_integer, \&is_integer)
    ->out(\&is_integer)
    ->enable;

sub surface {
    # no need to validate arguments anymore!
    # just implement the logic:
    return $_[0] * $_[1];
}
    \end{lstlisting}
\end{frame}

\begin{frame}
    \begin{center}
        Sub::Assert
    \end{center}
\end{frame}

\begin{frame}[fragile]
    \begin{lstlisting}
assert
    pre     => {
     'parameter larger than one' => '$PARAM[0] >= 1',
    },
    post    => '$VOID or $RETURN <= $PARAM[0]',
    sub     => 'squareroot',
    context => 'novoid',
    action  => 'carp';
    \end{lstlisting}
\end{frame}

\begin{frame}
    \begin{center}
        Carp::Datum
    \end{center}
\end{frame}

\begin{frame}[fragile]
    \begin{lstlisting}
use Carp::Datum;

sub routine {
    DFEATURE my $f_, "optional message";
    my ($a, $b) = @_;
    DREQUIRE $a > $b, "a > b";
    $a += 1; $b += 1;
    DASSERT $a > $b, "ordering a > b preserved";
    my $result = $b - $a;
    DENSURE $result < 0;
    return DVAL $result;
}
    \end{lstlisting}
\end{frame}

\begin{frame}
    \begin{center}
        MooseX::Contract
    \end{center}
\end{frame}

\begin{frame}[fragile]
    \begin{lstlisting}
    contract 'add'
        => accepts [ ...type... ]
        => returns void,
        with_context(
            pre => sub {
                ...
            },
            post => assert {
                ...
            }
        );
    sub add {
        my $self = shift;
        my $incr = shift;
        $self->{value} += $incr;
        return;
    }
    \end{lstlisting}
\end{frame}

\begin{frame}
    \begin{center}
        Attribute::Contract
    \end{center}
\end{frame}

\begin{frame}[fragile]
    \begin{lstlisting}
package AbstractRenderer;
use Attribute::Contract;

sub new {...}

sub render
  :ContractRequires(VALUE(Str), %ANY)
  :ContractEnsures(VALUE(Str))
  :ContractThrows(Exception::TemplateNotFound) {
}
    \end{lstlisting}
\end{frame}

\begin{frame}
    \begin{center}
        DbC и TDD
    \end{center}
\end{frame}

\begin{frame}
    \begin{center}
        DbC и Defensive programming
    \end{center}
\end{frame}

\begin{frame}
    \begin{center}
        Когда применять?

        \begin{itemize}
            \item Библиотеки, модули
            \item Интерфейсы
        \end{itemize}
    \end{center}
\end{frame}

\begin{frame}
    \begin{center}
        Как не применять?

        \begin{itemize}
            \item Не валидировать данные
            \item Не производить побочных эффектов
        \end{itemize}
    \end{center}
\end{frame}

\begin{frame}
    \begin{center}
        Спасибо!
    \end{center}
\end{frame}

\end{document}
