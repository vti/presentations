\documentclass[14pt]{beamer}
\usepackage{listings}
\usepackage[T2A]{fontenc}
\usepackage[utf8]{inputenc}
\usepackage[english,russian]{babel}

\begin{document}

\title{Pragmatic Perl}
\author{Вячеслав Тихановский (vti)}
\date{2013-12-21}
\setbeamercolor{background canvas}{bg=black,fg=white}
\setbeamercolor{frametitle}{fg=white}
\setbeamercolor{normal text}{bg=black}
\setbeamercolor{whitetext}{fg=white}
\usebeamercolor[fg]{whitetext}

\lstset{commentstyle=\textit}
\lstset{basicstyle=\small,language=Perl}

\frame{\titlepage}

\begin{frame}
    \begin{center}
        pragmaticperl.com
    \end{center}
\end{frame}

\begin{frame}
    \begin{center}
        Ежемесячный журнал о современном Perl!
    \end{center}
\end{frame}

\begin{frame}
    \frametitle{Desktop}
    \begin{center}
        \includegraphics<1>[height=8cm]{screenshot-desktop}
    \end{center}
\end{frame}

\begin{frame}
    \frametitle{Mobile}
    \begin{center}
        \includegraphics<1>[height=8cm]{screenshot-android}
    \end{center}
\end{frame}

\begin{frame}
    \frametitle{Books}
    \begin{center}
        PDF (A4, 9x12), MOBI, ePub, FB2
    \end{center}
\end{frame}

\begin{frame}
    \frametitle{Статистика}
    \begin{center}
        \pause

        \textbf{10} выпусков \pause

        \textbf{8} интервью \pause

        \textbf{13} авторов \pause

        \textbf{800} email-подписчиков \pause

        \textbf{350} rss-подписчиков
    \end{center}
\end{frame}

\begin{frame}
    \frametitle{Подписчики}
    \begin{center}
        \includegraphics<1>[height=6cm]{stat-subscribers}
    \end{center}
\end{frame}

\begin{frame}
    \frametitle{Посетители}
    \begin{center}
        \textbf{17 000} посетителей \pause

        \textbf{40 000} посещений \pause

        \textbf{16 000} скачиваний
    \end{center}
\end{frame}

\begin{frame}
    \frametitle{География}
    \begin{enumerate}
        \item Russia  58.84\%
        \item Ukraine   23.46\%
        \item Belarus   2.72\%
        \item Kazakhstan 1.77\%
        \item United States 1.71\%
        \item Netherlands 1.43\%
        \item Germany 1.12\%
        \item United Kingdom 0.73\%
        \item Israel 0.69\%
        \item France 0.47\%
    \end{enumerate}
\end{frame}

\begin{frame}
    \frametitle{Рефералы}
    \begin{enumerate}
        \item habrahabr.ru            33.13\%
        \item opennet.ru              25.36\%
        \item t.co                    6.40\%
        \item linux.org.ru            3.91\%
        \item vk.com                  3.71\%
        \item nixp.ru                 2.34\%
        \item linuxcenter.ru          2.04\%
        \item planetperl.ru           1.66\%
        \item ru-perl.livejournal.com 1.38\%
        \item blogs.perl.org          1.33\%
    \end{enumerate}
\end{frame}

\begin{frame}
    \begin{center}
        Как помочь?
    \end{center}
\end{frame}

\begin{frame}
    \begin{center}
        Подписаться \pause

        @PragmaticPerl \pause

        Поделиться \pause

        Написать статью
    \end{center}
\end{frame}

\begin{frame}
    \begin{center}
        Как журнал может вам помочь?
    \end{center}
\end{frame}

\begin{frame}
    \begin{center}
        Узнать что-то новое \pause

        Разместить вакансию \pause

        Анонсировать Perl-мероприятие \pause

        Рассказать о себе
    \end{center}
\end{frame}

\begin{frame}
    \begin{center}
        о-ло-ло
    \end{center}
\end{frame}

\begin{frame}
    \begin{center}
        --- Pragmatic Perl - первый в мире русскоязычный журнал, посвященный
        проблемам программирования на языке perl.
    \end{center}
\end{frame}

\begin{frame}
    \begin{center}
        --- Первые 8 страниц журнала похожи на «плач Ярославны» и попытку
        откопать Perl.
    \end{center}
\end{frame}

\begin{frame}
    \begin{center}
        --- Книга мёртвых.
    \end{center}
\end{frame}

\begin{frame}
    \begin{center}
        --- Посмотрим, протянет ли сей журнал дольше, чем «Практика
        функционального программирования».
    \end{center}
\end{frame}

\begin{frame}
    \begin{center}
        --- Вы серьёзно полагаете, что язык с закорючками перед именами
        переменных — инструмент профессионала?
    \end{center}
\end{frame}

\begin{frame}
    \begin{center}
        --- Сейчас и посмотрим сколько людей на перле кодят. Пока что 71.
    \end{center}
\end{frame}

\begin{frame}
    \begin{center}
        --- В хтмл версии очень нехватает выравнивания по ширине.
    \end{center}
\end{frame}

\begin{frame}
    \begin{center}
        --- Никому не нужны умные люди.
    \end{center}
\end{frame}

\begin{frame}
    \begin{center}
        --- Дайте Hello World уже для чайников, только чтобы он короче был чем
        на Пыхе.
    \end{center}
\end{frame}

\begin{frame}
    \begin{center}
        --- А как в perl с юникодом?
    \end{center}
\end{frame}

\begin{frame}
    \begin{center}
        --- Зачем Perl6 если уже пилят Perl7?
    \end{center}
\end{frame}

\begin{frame}
    \begin{center}
        --- Не хватает картинок. И не надо смеяться, я серьезно.
    \end{center}
\end{frame}

\begin{frame}
    \begin{center}
        --- Может вам накатать статью про хардкорный AnyEvent?
    \end{center}
\end{frame}

\begin{frame}
    \begin{center}
        Спасибо!
    \end{center}
\end{frame}

\end{document}
